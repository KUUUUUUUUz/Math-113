\documentclass{article}

\usepackage{amsmath, amsthm, amssymb, amsfonts}
\usepackage{thmtools}
\usepackage{graphicx}
\usepackage{setspace}
\usepackage{geometry}
\usepackage{float}
\usepackage{hyperref}
\usepackage[utf8]{inputenc}
\usepackage[english]{babel}
\usepackage{framed}
\usepackage[dvipsnames]{xcolor}
\usepackage{tcolorbox}

\definecolor{bubblegum}{rgb}{0.99, 0.76, 0.8}

\colorlet{LightGray}{White!90!Periwinkle}
\colorlet{LightOrange}{bubblegum!20}
\colorlet{LightGreen}{Green!15}
\colorlet{LightBlue}{Blue!15}
\colorlet{LightYellow}{Yellow!15}
\colorlet{LightRed}{Red!15}



\newcommand{\HRule}[1]{\rule{\linewidth}{#1}}

\declaretheoremstyle[name=Theorem,]{thmsty}
\declaretheorem[style=thmsty,numberwithin=section]{theorem}
\tcolorboxenvironment{theorem}{colback=LightGray}

\declaretheoremstyle[name=Proof,]{prosty}
\declaretheorem[style=prosty,numberlike=theorem]{Proof}
\tcolorboxenvironment{Proof}{colback=LightOrange}

\declaretheoremstyle[name=Example,]{prcpsty}
\declaretheorem[style=prcpsty,numberlike=theorem]{Example}
\tcolorboxenvironment{Example}{colback=LightGreen}

\declaretheoremstyle[name=Def]{prcpsty}
\declaretheorem[style=prcpsty,numberlike=theorem]{Def}
\tcolorboxenvironment{Def}{colback=LightRed}


\declaretheoremstyle[name=Problem,]{prcpsty}
\declaretheorem[style=prcpsty,numberlike=theorem]{Problem}
\tcolorboxenvironment{Problem}{colback=LightBlue}

\declaretheoremstyle[name=Solution,]{prcpsty}
\declaretheorem[style=prcpsty,numberlike=theorem]{Solution}
\tcolorboxenvironment{Solution}{colback=LightYellow}

\declaretheoremstyle[
    headfont=\bfseries\sffamily\color{NavyBlue!70!black}, bodyfont=\normalfont,
    mdframed={
        linewidth=2pt,
        rightline=false, topline=false, bottomline=false,
        linecolor=NavyBlue
    }
]{thmblueline}
\declaretheorem[style=thmblueline, numbered=no, name=Note]{note}

\setstretch{1.2}
\geometry{
    textheight=9in,
    textwidth=5.5in,
    top=1in,
    headheight=12pt,
    headsep=25pt,
    footskip=30pt
}
\newcommand\N{\ensuremath{\mathbb{N}}}
\newcommand\R{\ensuremath{\mathbb{R}}}
\newcommand\Z{\ensuremath{\mathbb{Z}}}
\renewcommand\O{\ensuremath{\emptyset}}
\newcommand\Q{\ensuremath{\mathbb{Q}}}
\newcommand\C{\ensuremath{\mathbb{C}}}
% ------------------------------------------------------------------------------

\begin{document}

% ------------------------------------------------------------------------------
% Cover Page and ToC
% ------------------------------------------------------------------------------

\title{ \normalsize \textsc{}
		\\ [2.0cm]
		\HRule{1.5pt} \\
		\LARGE \textbf{\uppercase{Abstract Algebra}
		\HRule{2.0pt} \\ [0.6cm] \LARGE{} \vspace*{10\baselineskip}}
		}
\date{}
\author{\textbf{Author} \\ 
		Persy \\}

\maketitle
\newpage

\tableofcontents
\newpage

% ------------------------------------------------------------------------------
\section{Groups}
% Unfinished, going to include Quiz and Homework questions, midterm questions, and some notes maybe
%

% ------------------------------------------------------------------------------
\section{Ring}


\subsection{Section 18: Ring & Fields}

\begin{theorem}
    This is a theorem.
\end{theorem}

\begin{Proof}
    this a proof
\end{Proof}

\begin{Example}
    This is a principle.
\end{Example}
\begin{Problem}
    This is a problem.
\end{Problem}

\begin{Solution}
    soln
\end{Solution}
\newpage

% ------------------------------------------------------------------------------
\subsection{Section 19: Integral Domains}

\begin{Example}
    Show$ \Z_p$ is a field when p is a prime number. 
\end{Example}
\begin{Proof}
    We only need to show that every nonzero element in $\Z_p$ is a unit.
    \\Take $a \in \Z_p. a\neq 0$ 
    \\Consider the map $\phi_a : \Z_p \rightarrow \Z_p $
    \\ \hspace*{3.5cm} $x \mapsto ax$ in $\Z_p$
    \\ We claim that $\phi_a $ is a bijection:
    \begin{itemize}
        \item $\phi_a$ is injective: $\phi_a(x) = \phi_a(y)$
        \\Then ax = ay $\Rightarrow$ $a(x-y) = 0$
        \\ Since $\Z_p$ is an integral domain and has no divisor of 0, then $a\neq0$
        \\ Thus $x = y$
    \end{itemize}
    \begin{itemize}
        \item $\phi_a$ is a surjective map since $\Z_p$ order is finite and injectivity implis surjectivity. \\ Then it is a bijective map.
    \end{itemize}
    \\Hence there is some $x\in \Z_p$ that $\phi_a(x) = 1$ which is $ax = 1$ and shows that a is a unit. 
\end{Proof}

\newpage

% ------------------------------------------------------------------------------
\subsection{Section 20: Fermat's Euler's theorems}

\begin{theorem}
    
    \textbf{Little Theorem of Fermat}
    \newline
    Let $a \in \Z$ p is a prime number. $p \nmid a.$
    Then $a^{p-1} \equiv$ 1 mod p
    \newline
    \textbf{Cor:} $a^{p} \equiv$ a mod p
\end{theorem}
\begin{Example}
    what is $8^{97}$ in $\Z_1_3$ (order12 Field)?
    \newline
    So $[8]^{12}=[1]$
    $97\div 12$ = 8
    \newline
    $8^{97}=8^{12*8+1}=([8]^{12})^8\cdot[8] = [1]^8\cdot[8]$
    \newline
    \newline
    \textbf{Using Cor:}
     $8^{97}=8^{13*7+12}=([8]^{13})^7\cdot([8]^{12}) = [8]^7\cdot[8]^{12} = [8]^7\cdot[1]$
     \newline
     $[8]\equiv[-5]$ mod 3
     \newline
     Thus:
     \newline
     $[-5]^7 = [-5]\cdot[-5]^6 = [-5]\cdot([-5]^2)^3 = [-5]\cdot([25]\equiv[-1])^3$
     
     \setlength\parindent{24pt} $= [-5][-1]$ = 5 mod 13
\end{Example}
\begin{Example}
    show that $15\mid(n^{33}-n)$ for all $a \in \Z$
    \newline
    Proof: 15 is not prime.
    \newline
    But $15 = 3\cdot5$
    \newline
    So it is enough to show that $3\mid (n^{33}-n)$ and $5\mid(n^{33}-n)$
    \newline
    We discuss by cases. 
    \begin{itemize}
        \item If $\textbf{3}\nmid n$ Then $n^{33} = (n^\textbf{2})^{16}\cdot n = 1\cdot n$ mod 3 (in Z3)
    \end{itemize}
    \hspace*{0.8cm} Thus $3\mid (n^{33} - n = 0 $ in $\Z_3$) 
    \begin{itemize}
        \item If $3\mid n$ Then $3\mid n\cdot (n^{32}-1)$
    \end{itemize}
    \hspace*{0.8cm} Thus $3\mid (n^{33}-n)$
    \\
    Similarly we show $5\mid(n^{33}-n)$
    \begin{itemize}
        \item If $\textbf{5}\nmid n$ Then $n^{33} = (n^\textbf{4})^8\cdot n = 1 \cdot n$ mod 5 (in Z5)
    \end{itemize}
    \hspace*{0.8cm} Thus $5 \mid (n^{33} - n = 0 $ in $\Z_5$) 
    \begin{itemize}
        \item If $5\mid n$ Then $5\mid n\cdot (n^{32}-1)$
    \end{itemize}
    \hspace*{0.8cm} Thus $5\mid (n^{33}-n)$
\end{Example}
\begin{note}
    $p1 < p2 < \cdots < pk$
    \\
    let m = $c(p1-1)(p2-1)\cdots (pk-1) + 1$ where c is some constant.
    \\
    $\Rightarrow p1 p2\cdots pk \mid$ $n^m -n$
\end{note}
% ------------------------------------------------------------------------------
\newpage
\begin{Def}
    Euler's generalization:
    \\
    \Z_n ^\times = {$m \in \Z_n \mid $ m is a unit}
    \\
    \hspace*{0.4cm} = ${m \in \Z_n \mid gcd(m,n) = 1}$
\end{Def}


\begin{Proof}
     $gcd(m,n) = 1 \Leftrightarrow$ m is not a divisor of zero (*)
    \\ $\Rightarrow$ Assume we know gcd(m, n) = 1 and to show m is a unit.
    \\ Thus for such m, construct $\phi_m: \Z_n \rightarrow \Z_n$
    \\\hspace*{5.0cm} $a \mapsto ma$
    \\By previous proof, we know that such map is bijection for $\Z_p$ And bascially we generalize it to take everything that is coprime with n. So we know that m must be a unit.
    \\ $\Leftarrow$ Assume we know it is a unit to show gcd(m,n) = 1
    \\Conversely, m is a unit imlies m is not a zero divisor, thus gcd(m,n) = 1
\end{Proof}
\begin{Def}
    Euler Phi-Function:
    $\phi (n)$ = \#\{$ m \in \Z ^+ \mid m \leq n$ gcd(m,n) = 1\}
    
\end{Def}

\newpage













\newpage
\subsection{Section 21: The Field of Fractions of Integral Domain}


\newpage
\subsection{Section 22: Rings of Polynomials}

\newpage



\newpage

% ------------------------------------------------------------------------------
% Reference and Cited Works
% ------------------------------------------------------------------------------

\bibliographystyle{IEEEtran}
\bibliography{References.bib}

% ------------------------------------------------------------------------------

\end{document}
