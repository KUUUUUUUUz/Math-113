
\documentclass{article}

\usepackage{amsmath, amsthm, amssymb, amsfonts}
\usepackage{thmtools}
\usepackage{graphicx}
\usepackage{setspace}
\usepackage{geometry}
\usepackage{float}
\usepackage{hyperref}
\usepackage[utf8]{inputenc}
\usepackage[english]{babel}
\usepackage{framed}
\usepackage[dvipsnames]{xcolor}
\usepackage{tcolorbox}

\definecolor{bubblegum}{rgb}{0.99, 0.76, 0.8}

\colorlet{LightGray}{White!90!Periwinkle}
\colorlet{LightOrange}{bubblegum!20}
\colorlet{LightGreen}{Green!15}
\colorlet{LightBlue}{Blue!5}
\colorlet{LightYellow}{Yellow!15}
\colorlet{LightRed}{Red!15}



\newcommand{\HRule}[1]{\rule{\linewidth}{#1}}

\declaretheoremstyle[name=Theorem,]{thmsty}
\declaretheorem[style=thmsty,numberwithin=section]{theorem}
\tcolorboxenvironment{theorem}{colback=LightGray}

\declaretheoremstyle[name=Proof,]{prosty}
\declaretheorem[style=prosty,numberlike=theorem]{Proof}
\tcolorboxenvironment{Proof}{colback=LightOrange}

\declaretheoremstyle[name=Example,]{prcpsty}
\declaretheorem[style=prcpsty,numberlike=theorem]{Example}
\tcolorboxenvironment{Example}{colback=LightGreen}

\declaretheoremstyle[name=Def]{prcpsty}
\declaretheorem[style=prcpsty,numberlike=theorem]{Def}
\tcolorboxenvironment{Def}{colback=LightYellow}


\declaretheoremstyle[name=Problem,]{prcpsty}
\declaretheorem[style=prcpsty,numberlike=no]{Problem}
\tcolorboxenvironment{Problem}{colback=LightGreen}

\declaretheoremstyle[name=Solution,]{prcpsty}
\declaretheorem[style=prcpsty,numberlike=no]{Solution}
\tcolorboxenvironment{Solution}{colback=LightBlue}

\declaretheoremstyle[
    headfont=\bfseries\sffamily\color{NavyBlue!70!black}, bodyfont=\normalfont,
    mdframed={
        linewidth=2pt,
        rightline=false, topline=false, bottomline=false,
        linecolor=NavyBlue
    }
]{thmblueline}
\declaretheorem[style=thmblueline, numbered=no, name=Note]{note}



\setstretch{1.2}
\geometry{
    textheight=9in,
    textwidth=5.5in,
    top=1in,
    headheight=12pt,
    headsep=25pt,
    footskip=30pt
}
\newcommand\N{\ensuremath{\mathbb{N}}}
\newcommand\R{\ensuremath{\mathbb{R}}}
\newcommand\Z{\ensuremath{\mathbb{Z}}}
\renewcommand\O{\ensuremath{\emptyset}}
\newcommand\Q{\ensuremath{\mathbb{Q}}}
\newcommand\C{\ensuremath{\mathbb{C}}}

\begin{document}



\title{ \normalsize \textsc{}
		\\ [2.0cm]
		\HRule{1.5pt} \\
		\LARGE \textbf{\uppercase{Abstract Algebra Problem set}
		\HRule{2.0pt} \\ [0.6cm] \LARGE{} \vspace*{10\baselineskip}}
		}
\date{}
\author{\textbf{Author} \\ 
		Persy \\}

\maketitle
\newpage

\tableofcontents
\newpage





% ------------------------------------------------------------------------------

\section{Sec 22 and 23}
\begin{Problem}
    Section 22.Problem 27
\end{Problem}
\begin{Solution}
    
\end{Solution}

\begin{Problem}
    Section 22.Problem 31
\end{Problem}

\begin{Solution}
    
\end{Solution}
\newpage
% ------------------------------------------------------------------------------

\section{Sec 18 and 19}
\begin{Problem}
    \textbf{Section 18. Problem 23.}
    Describe all homomorphism from $\Z \rightarrow \Z$
\end{Problem}

\begin{Solution}
    There are infinite group homomorphism. 
    \\ Look at $\phi(ab) \phi(a) * \phi(b)$
    \\ $\phi(1) = n$
    \\  $\phi(1*a)  = \phi(a) =  \phi(1) + .. \phi(1) = a*n = an$
    \\ $\phi(1*a)  = \phi(a) * \phi(1) = n * an$ Thus must have $an = nan$
    \\ Thus either $\phi(1) = 1 $ or $\phi(1) = 0$
\end{Solution}

\begin{Problem}
\textbf{Section 18 Problem 24 and 25}
    (1) Describe all ring homomorphisms of $\Z$ onto $\Z \times \Z$
    \\(2) Describe all  all ring homomorphisms $\Z \times \Z$
\end{Problem}

\begin{Solution}
    \textbf{(1) }Similar reasoning. 
    \\$\phi(1) = (0,0)$, $\phi(1) = (1,0)$, $\phi(1) = (1,1)$, $\phi(1) = (0,1)$

    \textbf{(2)} The only well defined maps are:
    \\$\phi(m,n) = m$,$\phi(m,n) = n$, $\phi(m,n) = n+m$, $\phi(m,n) = 0$
    \\ The first two are checked in Q23. The last is trivial homomorphism. 
    \\ We can construct a counter example for the third. Thus only 3. 
\end{Solution}
\begin{Problem}
    \textbf{Section 18. Problem 38.}
    Show that $a^2 - b^2 = (a+b)(a-b)$ for all a and b in ring R if and only if R is commutative. 
\end{Problem}

\begin{Solution}
Consider $(a+b)(a-b) = a^2 -ab + ba -b^2.$ 
\\Commutative implies ab = ba.
\end{Solution}

\begin{Problem}
\textbf{Section18: Problem 41}
    An element a of a ring R is nilpotent if $a^n = 0 $ for some $n \in \Z^+$ Show that a and b are nilpotent elems of a comuutative ring, then $a+b $ also nilpotent.
\end{Problem}

\begin{Solution}
    
\end{Solution}


\begin{Problem}
    \textbf{Section 18: Problem 55}
    A ring a boolean ring if $a^2 = a$ for all$ a \in R$ so that every element is idempotent. Show that every boolean ring is commutative. 
\end{Problem}



\begin{Solution}
    
\end{Solution}


\begin{Problem}
    \textbf{Section 19: Probelm 24}
    \\Show that an intersection of subdomains of an integral domain D is again a subdomain D. 
\end{Problem}
\begin{Solution}
    Solution: Sub-integral domain share unity. -----unf
\end{Solution}


% ------------------------------------------------------------------------------


\section{Sec 15}

\begin{Problem}
    Classify the group $(\Z_4 \times \Z_4)/<(1,2)>$
\end{Problem}
\begin{Solution}
    Solution: $<(1,2)> = (1,2), (2,0),(3,2),(0,0)$ order 4.
    \\Ord($\frac{16}{4}$)= Ord(4). Order 4 groups are only $\Z_2\times\Z_2$ or $\Z_4$. Thus we check if our quotient group has order 4 elements.
    \\Take element $(1,1) +<(1,2)> \Rightarrow 4(1,1) +<(1,2)> = <(1,2)>$ which is order 4. 
\end{Solution}
\end{document}
% ------------------------------------------------------------------------------

